
\documentclass{isimabeamer}


%%%%%%%%%%%%%%%%%%%%%%%%%%%%%%%%%%%%%%%%%%%%%%%%

\AtBeginSection[]{
  \begin{frame}
  \vfill
  \centering
  \begin{beamercolorbox}[sep=8pt,center,shadow=true,rounded=true]{title}
    \usebeamerfont{title}\insertsectionhead\par%
  \end{beamercolorbox}
  \vfill
  \end{frame}
}

\title[Théorie des jeux: les jeux coopératifs]{Théorie des jeux: les jeux coopératifs}
% Titre du diaporama

\subtitle{application aux réseaux de communication}
% Sous-titre optionnel

\author{Rick Sanchez\inst{1}, Morty Smith\inst{2}}
% La commande \inst{...} Permet d'afficher l' affiliation de l'intervenant.
% Si il y a plusieurs intervenants: Marcel Dupont\inst{1}, Roger Durand\inst{2}
% Il suffit alors d'ajouter un autre institut sur le modèle ci-dessous.

\institute[ISIMA]
{
  \inst{1}%
  Institut Supérieur d'Informatique, de Modélisation et de leurs Applications
  \inst{2}%
  Harry Herpson High School

 }


\date{19 décembre 2018}
% Optionnel. La date, généralement celle du jour de la conférence

\subject{Recette de la matière noire concentrée}
% C'est utilisé dans les métadonnes du PDF



\logo{
\includegraphics[scale=0.20]{images/isima.png}
}



%%%%%%%%%%%%%%%%%%%%%%%%%%%%%%%%%%%%%%%%%%%%%%%%%%%%%%%%%%%%%%%%%%%%%
\begin{document}

\begin{frame}
  \titlepage
\end{frame}



\begin{frame}{Sommaire}
  \tableofcontents
  % possibilité d'ajouter l'option [pausesections]
\end{frame}


\section{Introduction}

\subsection{Contexte}
\begin{frame}{Contexte}

\begin{block}{Rappels théorie des jeux}
\begin{itemize}
    \item Théorie des jeux: cadre d'étude des interactions entre agents/joueurs
    \item Nombreuses applications: ingénierie, psychologie, sociologie, politique etc. 
    \item Deux branches principales: 
    \begin{itemize}
        \item non-coopératif: concurrence entre agents
        \item \textbf{coopératif}: étude des comportements lorsque les agents coopèrent 
    \end{itemize}
    \end{itemize}
\end{block}

\begin{block}{La théorie des jeux dans les réseaux de communication}
\begin{itemize}
    \item En 2009 (date de l'article) surtout non-coopératif
    \item \textbf{Mais} plusieurs articles démontrant que coopération = possibilités d'amélioration dans les couches basses du modèle OSI.  
    \end{itemize}
\end{block}
\end{frame}

\subsection{Problématique}
\begin{frame}{Problématique}
\begin{block}{Problèmes}
\begin{itemize}
    \item Recherche actuelle limitée 
    \item Manque un cadre unifié pour appliquer jeux coopératifs aux réseaux 
\end{itemize}
\end{block}

\begin{block}{But de l'article}
\begin{itemize}
    \item Présenter la théorie des jeux collaboratifs 
    \item Proposer un cadre unifié sous forme de classification orientée ingénierie.
\end{itemize}
\end{block}
\end{frame}


\begin{frame}{Cadre proposé}
\begin{block}{Classe I - Jeux canoniques}
\begin{itemize}
    \item Structure optimale: tous les joueurs
    \item \textbf{Problématique}: Comment stabiliser?  
\end{itemize}
\end{block}

\begin{block}{Classe II - Jeux de formation de coalition}
\begin{itemize}
    \item Se rassembler a un coût  
    \item \textbf{Problématique}: Quelles structures vont se former | Comment les étudier?
\end{itemize}
\end{block}

\begin{block}{Classe III - Jeux de graphes de coalition}
\begin{itemize}
    \item Communication entre les joueurs sous forme de graphe
    \item \textbf{Problématique}: Comment stabiliser/créer une structure en prenant en compte le graphe de communication?
\end{itemize}
\end{block}

\end{frame}


\section{Jeux de coalition - Bases}

\begin{frame}{Bases}
\begin{block}{Bases}
\begin{itemize}
    \item Fait intervenir un ensemble de joueurs $N = \left \{1,...,n \right \}$ avec $\text{card}(N) = n$.
    \item Les joueurs cherchent à former des groupes pour améliorer leur score (gain++ ou perte--).
    \item Un groupe ou coalition $S \subseteq N $ représente un accord entre les joueurs de $S$ pour agir en tant qu'entité. 
\end{itemize}
\end{block}

\begin{block}{Bases - suite}
\begin{itemize}
    \item La valeur de coalition notée $v$ représente la valeur associée a une coalition.  
    \item Un jeu de coalition est défini par un couple $(N,v)$
\end{itemize}
\end{block}
\end{frame}




\section{Jeux de coalition canoniques}

\subsection{Propriétés}

\begin{frame}{Propriétés}
\begin{block}{Propriétés}
\begin{itemize}
    \item Jeu sous forme caractéristique (TU ou NTU)
    \item Super-additivité $v(S_1 \cup S_2) \ge v(S_1) + v(S_2)  \forall S_1 \subset  N, S_2 \subset N, S_1 \cap S_2 = \varnothing   $
\end{itemize}
\end{block}
\begin{block}{Objectif}
\begin{itemize}
\item Propriété et stabilité de la grande coalition
\item La distribution de gain d'une manière optimale
\end{itemize}
\end{block}

\end{frame}

\subsection{Aspects importants}
\begin{frame}{Aspects importants}
\begin{block}{Aspects importants}
\begin{itemize}
    \item L'allocation optimale de gain aux joueurs de la grande coalition
    \item L'assurance de l'équité dans le partage des gains de la grande coalition
\end{itemize}
\end{block}
\end{frame}

\subsection{Coeur / Core}
\begin{frame}{Coeur / Core}
\begin{block}{Définition (Coeur)}
\begin{itemize}
    \item Stabilité de la grande coalition
    
    $C_{TU} = \left \{ x: \sum_{i \in N} x_i = v (N) \text{ and } \sum_{i \in S} x_i \ge v(S) \forall S \subseteq N \right \}$
    
    \item L'assurance de l'équité dans le partage des gains de la grande coalition
\end{itemize}
\end{block}

\begin{block}{Propriété et existence}
\begin{itemize}
    \item L'existence n'est pas garantie
    \item Le coeur est vide $\rightarrow$ la coalition n'est pas stable    
\end{itemize}
\end{block}

\end{frame}



\subsection{Valeur de Shapley}
\begin{frame}{Valeur de Shapley}
\begin{block}{Axiomes}
\begin{enumerate}
    \item \textbf{Efficacité}: $\sum_{i \in N} \phi_i(v) = v(N)$.
    \item \textbf{Symétrie}: Si un joueur $i$ et un joueur $j$ sont tels que $v\left ( S \cup \left \{ i \right \} \right ) = v \left ( S \cup \left \{ j \right \} \right )$ pour toute coalition $S$ qui ne contient ni $i$ ni $j$, alors $\phi_i(v) = \phi_j(v)$. 
    \item \textbf{Dummy player}: Si un joueur $i$ est tel que $v(S) = v\left ( S \cup \left \{ i \right \} \right )$ pour toute coalition $S$ qui ne contient pas $i$, alors $\phi_i(v) = 0$.
    \item \textbf{Additivité}: Si $u$ et $v$ sont des fonctions caractéristique, alors $\phi(u+v) = \phi(u) + \phi(v)$.
\end{enumerate}

\end{block}


\end{frame}

\begin{frame}{Valeur de Shapley}
\begin{block}{Propriétés}
\begin{itemize}
    \item Solution unique pour un jeu $(N, v)$
\end{itemize}
\end{block}

\begin{block}{Formule de calcul}
\begin{itemize}
    \item $\phi_i(v) = \sum_{S \subseteq N \setminus \left \{ i \right \}} \frac{\left | S \right |!\left ( N- \left | S \right | -1 \right)!}{N!} \left [ v(S \cup \left \{ i \right \}) - v(S) \right ]$
\end{itemize}
\end{block}
\end{frame}


%% Nucleolus 

\subsection{Nucleolus}
\begin{frame}{Nucleolus}
\begin{block}{Nucleolus}
\begin{itemize}
    \item Un joueur donne l'allocation minimisant l'insatisfaction des joueurs par rapport à l'allocation reçue pour un jeu $(N,v)$.  
    \item La mesure de l'insatisfaction pour une allocation $x \in \mathbb{R}^{|N|}$ est l'excès: 
    
    $e(x, S) = v(S) - \sum_{j\in S} x_j$
    
\end{itemize}
\end{block}
\end{frame}


\subsection{Rate allocation in a multiple access channel}
\begin{frame}{Rate allocation in a multiple access channel}
\begin{block}{Contexte}
Répartition équitable des taux de transmission entre plusieurs utilisateurs d'un réseau sans fil utilisant un canal MAC gaussien.
\end{block}
\begin{block}{Modèle: Jeu de coopération $(N,v)$}
\begin{itemize}
  \item $N = {1,...,N}$ joueurs, utilisateurs souhaitant accéder au canal.
        \item $v$ la somme maximale du taux de transmission qu'une coalition S peut atteindre
\end{itemize}

\end{block}

\begin{block}{Propriétés}
Super-additivité 
\end{block}

\end{frame}



\begin{frame}{Rate allocation in a multiple access channel}
\begin{block}{Objectif}
L'allocation des capacités pour la grande coalition $v(N)$ d'une manière équitable:
\begin{itemize}
    \item Le coeur / core
    \item Valeur de Shapeley
\end{itemize}
\end{block}
\end{frame}




% ====== FIN Asmae - Jeux canoniques =====





\section{Jeux de formation de coalition}
\begin{frame}{Propriétés}

\begin{block}{Propriétés}
\begin{itemize}
    \item Pas super additif <=> grande coalition pas toujours avantageux
    \item Coût lié à la coopération
    \item Plusieurs coalitions en fonction gain/coût
    \item Structure peut changer au cours du temps (agents quittent /rejoignent d'autres coalitions en fonction du jeu)
    \item Plusieurs méthodes de résolution, plus complexe que canoniques et dépendant de la situation étudiée
\end{itemize}
\end{block}

\begin{block}{Applications}
\begin{itemize}
    \item Négociation/échange d'information pour former une coalition 
    \item Exemple mining de cryptomonnaies comme Bitcoin
    \item Plusieurs coalitions en fonction gain/coût
\end{itemize}
\end{block}

\end{frame}


\section{Jeux de graphes de coalition}
\begin{frame}
\begin{block}{Propriétés}
\begin{itemize}
    \item Valeur coalition ne dépend pas uniquement des agents qui la composent
    \item Importance de comment les agents sont inter-connectés
    \item Impact important sur le jeu
    \item Comment former le graphe => problème complexe (explosion combinatoire)
    \item Plusieurs méthodes de résolution, basées notamment sur les jeux non-collaboratifs et des algorithmes stochastiques (solutions approchées)
\end{itemize}
\end{block}

\begin{block}{Applications}
\begin{itemize}
    \item Internet/Réseaux locaux sous forme de graphes 
    \item Design de la dispositions des antennes relais dans les réseaux sans-fil
\end{itemize}
\end{block}
\end{frame}




\section{Conclusion}

\begin{frame}{Conclusion}
\begin{block}{Conclusion}
\begin{itemize}
    \item Vue d'ensemble des jeux de coalition
    \item Nombreuses applications notamment \textbf{réseaux} 
    \item Proposition de classification
    \begin{itemize}
        \item \textbf{Canoniques}
        \item Formation de coalitions
        \item Graphes de coalitions
    \end{itemize}
\end{itemize}
\end{block}
\end{frame}




\end{document}
